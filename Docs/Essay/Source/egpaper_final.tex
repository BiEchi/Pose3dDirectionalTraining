\documentclass[10pt,twocolumn,letterpaper]{article}

\usepackage{cvpr}
\usepackage{times}
\usepackage{epsfig}
\usepackage{graphicx}
\usepackage{amsmath}
\usepackage{amssymb}

% Include other packages here, before hyperref.

% If you comment hyperref and then uncomment it, you should delete
% egpaper.aux before re-running latex.  (Or just hit 'q' on the first latex
% run, let it finish, and you should be clear).
\usepackage[breaklinks=true,bookmarks=false]{hyperref}

\cvprfinalcopy % *** Uncomment this line for the final submission

\def\cvprPaperID{****} % *** Enter the CVPR Paper ID here
\def\httilde{\mbox{\tt\raisebox{-.5ex}{\symbol{126}}}}

% Pages are numbered in submission mode, and unnumbered in camera-ready
%\ifcvprfinal\pagestyle{empty}\fi
\setcounter{page}{1}
\begin{document}

%%%%%%%%% TITLE
\title{A Directional Training Method Based on \textit{VideoPose3D}}

\author{Hao Bai\\
ZJU-UIUC Institute\\
Haining, Jiaxing\\
{\tt\small https://jackgetup.com}
% For a paper whose authors are all at the same institution,
% omit the following lines up until the closing ``}''.
% Additional authors and addresses can be added with ``\and'',
% just like the second author.
% To save space, use either the email address or home page, not both
\and
\textbf{Prof.} Gaoang Wang\\
ZJU-UIUC Institute\\
Haining, Jiaxing\\
{\tt\small gaoangwang@intl.zju.edu.cn}
}

\maketitle
%\thispagestyle{empty}

%%%%%%%%% ABSTRACT
\begin{abstract}
   In this essay we perform a directional training method based on \textit{VideoPose3D}. 
   The original paper utilizes a semi-supervised approach on different kinds of subjects
   in the network, and we select each 
\end{abstract}

%%%%%%%%% BODY TEXT
\section{Introduction}

Our work focuses on the improvement of 3D human pose estimation in video.
\textit{VideoPose3D} is the current state-of-the-art approach which formulate
the problem as 2D keypoint detection followed by 3D pose estimation.


%-------------------------------------------------------------------------
\subsection{Language}

All manuscripts must be in English.

%------------------------------------------------------------------------
\section{Related Work}

All text must be in a two-column format. The total allowable width of the
text area is $6\frac78$ inches (17.5 cm) wide by $8\frac78$ inches (22.54
cm) high. Columns are to be $3\frac14$ inches (8.25 cm) wide, with a
$\frac{5}{16}$ inch (0.8 cm) space between them. The main title (on the
first page) should begin 1.0 inch (2.54 cm) from the top edge of the
page. The second and following pages should begin 1.0 inch (2.54 cm) from
the top edge. On all pages, the bottom margin should be 1-1/8 inches (2.86
cm) from the bottom edge of the page for $8.5 \times 11$-inch paper; for A4
paper, approximately 1-5/8 inches (4.13 cm) from the bottom edge of the
page.

%-------------------------------------------------------------------------
\subsection{Margins and page numbering}

All printed material, including text, illustrations, and charts, must be kept
within a print area 6-7/8 inches (17.5 cm) wide by 8-7/8 inches (22.54 cm)
high.
Page numbers should be in footer with page numbers, centered and .75
inches from the bottom of the page and make it start at the correct page
number rather than the 4321 in the example.  To do this fine the line (around
line 23)
\begin{verbatim}
%\ifcvprfinal\pagestyle{empty}\fi
\setcounter{page}{4321}
\end{verbatim}
where the number 4321 is your assigned starting page.

Make sure the first page is numbered by commenting out the first page being
empty on line 46
\begin{verbatim}
%\thispagestyle{empty}
\end{verbatim}


%------------------------------------------------------------------------
\section{Experimental Setup}

You must include your signed IEEE copyright release form when you submit
your finished paper. We MUST have this form before your paper can be
published in the proceedings.

Please direct any questions to the production editor in charge of these 
proceedings at the IEEE Computer Society Press: 
\url{https://www.computer.org/about/contact}. 


\section{Results}


{\small
\bibliographystyle{ieee_fullname}
\bibliography{egbib}
}

\end{document}
